\documentclass[answers,a4paper,11pt]{exam}
\usepackage{hyperref}
\ifx\pdftexversion\undefined
\usepackage{epsfig}
\else
\usepackage[pdftex]{graphics}
\fi
\usepackage{epsfig}

\begin{document}
\extraheadheight{.5in}
\firstpageheader{\large\sf FMC1203}%
{\large\sf National University of Singapore\\ School of Computing \\
\LARGE\sf Recursion}%
{\large\sf Semester 2 11/12}
\firstpageheadrule
\pagestyle{headandfoot}

\begin{questions}
\question
In computer graphics, nature scenes can often be generated recursively.  Find out how mountains can be generated with a recursive procedure.
\fillwithdottedlines{5in}
\newpage

\question
An anagram of a phrase is a rearrangement of non-space characters in the phrase to form another string consisting of words from a dictionary.  For example, an anagram for "Mr. Ooi Wei Tsang" is "Two Same Origin".

You are given a phrase and a dictionary of words.  Your task is to find all anagrams of the phrase with words from the dictionary.  Explain how you would solve the task, using the help of recursion.

%See \url{http://www.ssynth.co.uk/~gay/anagott.html}
\fillwithdottedlines{5in}
\newpage
\question

Bucket fill is a tool that is commonly used in painting application to completely paint a region (bounded by another color) with a given color.

How would you implement bucket fill recursively?

\fillwithdottedlines{5in}
\newpage
\question

Ray Tracing is a recursive technique to generate realistic synthetic images.  The technique is recursive in nature.  How does ray tracing work?

\begin{figure}[h!]
\begin{center}
\includegraphics[scale=0.3]{glasses.jpg}
\caption{"Glasses" by Gilles Tran (2006)}
\end{center}
\end{figure}

\fillwithdottedlines{4in}
\end{questions}

\end{document}
