\documentclass[answers,a4paper,11pt]{exam}
\usepackage{hyperref}
\ifx\pdftexversion\undefined
\usepackage{epsfig}
\else
\usepackage[pdftex]{graphics}
\fi
\usepackage{epsfig}

\begin{document}
\extraheadheight{.5in}
\firstpageheader{\large\sf FMC1203}%
{\large\sf National University of Singapore\\ School of Computing \\
\LARGE\sf Parallel and Distributed Computing}%
{\large\sf Semester 2 11/12}
\firstpageheadrule
\pagestyle{headandfoot}
In scenarios where it becomes ineffecient or inadequate to store the data in one place and execute a task in one place, one can turn to parallel and distributed computing as a solution.

\begin{questions}
\question
In task parallelism, many tasks (usually the same kind) are carried out at the same time in different places.  The task results may have to be collated at the end.

In data parallelism, the data are stored into, or retrieved from, different places at the same time.

Give a few examples of task parallelism and data parallelism in both computing and non-computing domain.
\fillwithdottedlines{3in}

\question
What are some considerations that you have to be aware of before you parallelize a task?
\fillwithdottedlines{3in}

\newpage
\question
In distributed computing, a task is divided into two or more subtasks, and is carried out in different places.  Data can also be stored in, and retrieved from, different places.

Give a few examples of distributed task computation and distributed data storage in both computing and non-computing domain.
\fillwithdottedlines{3in}

\question
What are some considerations that you have to be aware of before you distribute data or task to different places?
\fillwithdottedlines{3in}

\end{questions}
\end{document}
